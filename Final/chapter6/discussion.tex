\documentclass[../main.tex]{subfiles}


\begin{document}
\raggedright
Looking at the project from the very beginning, creating a web-based system over a smart application or block-chain was a good decision because everything that a smart application can achieve has been achieved with the web-based system and more such as a mobile first code and accessibility. Block-chain, on the other hand, would have been a good base for development but would not be highly maintainable or confidential even though it would be secure and would prevent data losses. Ruby on Rails or Laravel were also good options to use when producing this system as they would produce a similar system with a few distinct conditions. It has been easier to use Django compared to Ruby on Rails, based on past experience, as it allows a much easy platform to code with. Ruby on Rails has \textit{GEMS} that can be used, but Django allows usage of packages created by other users who publish their code under the MIT or Public licenses. An example of such a package would be All-Auth which is highly used and extremely secure due to continuous maintenance of the code. \\[4mm]

Personally, it has been a great learning experience with Django. Even though having used various other web-based coding platforms such as NodeJS, Ruby on Rails, a little PHP; Django has been the easiest to work with not only because it follows a simple model, view and controller system but because it allowed creating database schemas just as easily as creating an input form in HTML. The same database schemas can then be used as forms to receive data from the website. This makes production extremely easy and quick compared to other systems where each attribute needs to be defined. Django also provided an easy way to debug the code without having to figure out exactly what went wrong in a huge pile of code.  \\[4mm]

NodeJS would have been a good alternate option as it would allow creating a progressive web application with the ability to work offline. This would behave like a smartphone application. However, the security of such a system would be a high concern as using a lot of JavaScript can allow Cross-Site Scripting attacks and unlike Django, it would be difficult to protect against such attacks. 
\end{document}
