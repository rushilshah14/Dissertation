\documentclass[../main.tex]{subfiles}


\begin{document}
\raggedright
The current system can be the base for an entirely secure and efficient future system. The following is a list of possible implementations which can be completed by further studies. Each states the difficulty of producing it as well as if the code would be maintainable. \\[4mm]

Use of Django file encryption projects\cite{ruddra}\cite{danielquinn} to create a custom encrypted file upload system which stores the uploaded files with a \textit{SHA256 encryption} onto the hosting servers folder system. Both projects are not maintained and hence would require re-coding the whole encryption system by using one of them as the base and keeping it up to date at all times due to security. Producing something like this would require encryption expertise as well as time and research. It would be highly maintainable as 40 security flaws are detected each day based on a 2017 survey\cite{vulner}. File encryption is important because in case an unauthorised user gains access to the hosting server of the system, they can access the form files directly and view all the data. \\[4mm]
  
Django-pgcrypto\cite{dbencrypt} can be implemented onto the \textit{PostgreSQL} database server in order to encrypt every single entry into the database which will protect against any unauthorised access directly into the database. This system would work with the current \textit{forms.py} and encrypt data as it is entered into the database. Again in a scenario when an unauthorised user gains access to the database directly, in order to view the data inside, they would need to decrypt each of the fields making it extremely difficult to crack and hence making the system secure. Django-pgcrypto\cite{dbencrypt} can be used as the base of creating an encryption system specific to extenuating circumstances making it highly difficult and maintainable as you would need to keep the encryption up to date. \\[4mm]
  
Code to allow the secretary and student to communicate. A messaging system as such would allow both the secretary and the student to converse within the form. Producing such would require knowledge on \textit{WebRTC} and would not need to be highly maintained, once the code has been placed it would not require much attention unless a security bug is detected. A messaging system was not part of the requirements and therefore was left out but as a few students recommended it would be highly useful. \\[4mm]
  
A text area within the form accessible by the secretary and the scrutiny panel on writing down the minutes of the meeting. This would be very useful as the minutes can be referred to directly when accessing the form. However, these would be hidden from everyone else including the staff members and the student. Producing this would require an editor inbuilt with the \textit{textarea} box and the need to \textit{autoescape} the data filled in. Once the code is developed, it would not require much maintenance as long as the code is secure. 

\subsection*{Alternate uses for such a system}
This project has been focused on security, ease of access and the ability to get data off users and accessed by other users. A system like this can be edited into various other branches which involve interaction between two groups of users who require confidentiality and security. Below is a list of possible uses and scenarios where such a system can be highly useful if edited to the needs.\\[4mm]
 
Survey System - Gathering information with questions being asked in place of the extenuating circumstance form and assessed by the supervisors. Such a system can be edited to fill the form and submit without having the user to log in or provide personal details. Statistics can be added including graphs to calculate the individual results with the use of \textit{Python} and \textit{NPM}. \\[4mm]

Grading System - Users can upload files and data similarly to the current Extenuating Circumstance Form, and a superior user can assess it and graded accordingly. To achieve this, the system would need to duplicate the status system(Pending, Approved, Rejected) and allow it to grade based on integers for example; \textit{80/100}. It can be used in all kinds of grading scenarios such as education, work even personal data storage with grades. For example, a student can decide to store all their assignment on the system as well as the grade they have achieved. \\[4mm]

Medical Appointments \& Meetings - The system can be adjusted to allow patients to fill out medical forms directly onto the system which are then assessed and viewed by superiors. A payment system can also be directly added onto the system which allows the patient to have records of their payment and each medical form they have filled or to record every meeting occurred between the two. With further implementations, each meeting can have its minutes recorded onto the system.\\[4mm]

These are just a few examples of where such a system can be useful. With more encryption as mentioned under Chapter \ref{ch:futurework} this system can be developed into very many different branches of data retrieval.
\end{document}
