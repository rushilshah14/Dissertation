\documentclass[../main.tex]{subfiles}


\begin{document}
\raggedright
The interface will consist of two different types of users; The Students and the Scrutiny Committee. There will be other third party users who will get the data from the student records such as the professors and personal tutors. 

\subsection{Student Panel}
The student panel will consist of three main components; They should be able to submit a new circumstance form, access any previously submitted form and lastly check the status(described under Scrutiny Committee Panel) for any new feedbacks or requests from the scrutiny committee and upload a response or evidence as required. As stated in the previous section, the students should be able to login with their own university accounts making is easy to access from within the portal.

\subsection{Scrutiny Committee Panel}
This panel will be a little more detailed than the student panel. The panel would notify the scrutiny committee by email when a new form has been received The committee would be able to view the form and request any new feedback or evidence and change the status of the form. The status could be "Approved" or "Requires changes" or "Pending" which would allow different members of the scrutiny committee to know the status of a form. In this case;
\begin{itemize}
\item Approved would mean the student does not need to provide more information or evidence and the details of the form have been passed on to the third parties.
\item Requires changes would mean the student has to provide more information such as evidence or detailed writing. The requirements would be accessible by the student under the "Feedback" section of the form.
\item Pending would mean the form has not yet been checked by the members of the scrutiny committee and so there is nothing to be done. 
\end{itemize}
They can also export the same form as a PDF and download it for printing purposes or for documentation. \\[2mm]

	\begin{figure}[H]
        \center{\includegraphics[scale=0.4]
        {images/portalandfeedback.png}}
        \caption{\label{fig:formandfeedback} Form with a feedback section}
      \end{figure}
      
Figure~\ref{fig:formandfeedback} shows the portal with a form being displayed with a feedback and comments section. This section will allow communication between the student and the scrutiny committee as well as allow uploads of new evidence. 

The panel will also have a list of all the students who have submitted a form allowing the scrutiny committee to sort and search through the records to find any specific details they like which would be useful in cases of a review.\\[2mm]

It will also enable the exporting of data to third parties, only approved details by the scrutiny committee will be accessible by the third parties that need access and knowledge of an existing extenuating circumstance form.
  
\end{document}
