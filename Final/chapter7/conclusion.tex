\documentclass[../main.tex]{subfiles}


\begin{document}
\raggedright
The overall goal of the project was to successfully create a system which allows the students to submit extenuating circumstance forms and the secretary/scrutiny committee to have access to it without having to go through the troubles of multiple paper forms creating confusion. Being that the circumstances collect personal data, security was a big concern to digitisation. Throughout the documentation from Chapter 2 till Chapter 6, the requirements has been the sole purpose of decisions made. Django, even though it is a web based framework, provided us with high end security control without having to write down huge amounts of code for data safety. The system also provides flexibility such that it can use both \textit{SQLite} and \textit{PostgreSQL} databases to store the data, with the database being password protected. \\[4mm]

Secure authentication is present in the current system, if the system is hosted under the University's Intranet, it would have an extra authentication forcing only access by students and staff. This would be extremely useful for the Student Records page which contains a sanitised version of the circumstances, however, even without the extra authentication a secret key system would prevent unauthorised access as described under Chapter 6. \\[4mm]

With future work on this system it can be made extremely secure not only for University of Sheffield's extenuating circumstances but also for other system which require communication between users and submission of forms as discussed in Chapter 6.4. \\[4mm]

To condlude, developing this system has been an interesting journey, one of the key aspects was understanding a completely new system and its abilities; how similar yet different technologies such as Django, Laravel and Ruby on Rails can provide different back-end results yet produce the same front-end system as described under Chapter 2. Bundling up all the requirements from (Chapter 3) with the research (Chapter 2) and the initial design (Chapter 4) of the system, the final implementation has come really close to being entirely completed. With the addition of two encryptions(Chapter 6.4), one for each entry in the database and second for each file uploaded, this system can be complete while fulfilling all the requirements as well as adding additional features which serve to be useful. Deploying the code on \textit{Heroku} is proof that the implemented system is fully functional.
\end{document}
