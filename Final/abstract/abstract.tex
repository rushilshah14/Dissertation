\documentclass[../main.tex]{subfiles}


\begin{document}
\raggedright
This project manages the extenuating circumstances process by eliminating paper-work and increasing efficiency. Django, a Python framework, provided a platform to create a web-based system allowing students to submit their problems and circumstances on a portal. The scrutiny committee and the secretary can also access these forms and decide if the student gets approved/rejected for leniency toward their academics or needs to upload additional proof. The secretary fills out a sanitised version of the form which can be accessed with unique links and shared on the student's portfolio. In situations where a review is called for, the data can be retrieved, filtered, printed or downloaded from the portal. The system not only allows data to be organised but also accessible on any device as long it has internet. The code passed through different tests and scenarios in terms of security as well as functionality. Therefore, it has a decent element of security (provided by Django's inbuilt features and additional packages from other developers). However, with further work, the system can be made bulletproof in terms of security and have additional features which can make communication easier between all stakeholders. 

\end{document}
